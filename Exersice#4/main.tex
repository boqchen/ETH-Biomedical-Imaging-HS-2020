\documentclass[11pt,a4paper,BCOR12mm, headexclude, footexclude, twoside, openright]{scrartcl} 
\usepackage[scaled]{helvet}
\usepackage[british]{babel}
\usepackage[utf8]{inputenc}
\usepackage[T1]{fontenc}
\usepackage{fancyhdr}
\usepackage{lastpage}
\usepackage{ifthen}
\usepackage{amsmath,amsfonts,amsthm}
\usepackage{sfmath}
\usepackage{makecell}
\usepackage{booktabs}
\usepackage{sectsty}
\usepackage{graphicx}
\usepackage{amssymb}

%\KOMAoptions{optionenliste}
%\KOMAoptions{Option}{Werteliste}


\addtokomafont{caption}{\small}
%\setkomafont{descriptionlabel}{\normalfont
%	\bfseries}
\setkomafont{captionlabel}{\normalfont
	\bfseries}
\let\oldtabular\tabular
\renewcommand{\tabular}{\sffamily\oldtabular}
\KOMAoptions{abstract=true}
%\setkomafont{footnote}{\sffamily}
%\KOMAoptions{twoside=true}
%\KOMAoptions{headsepline=true}
%\KOMAoptions{footsepline=true}
\renewcommand\familydefault{\sfdefault}
\renewcommand{\arraystretch}{1.1}
\newcommand{\horrule}[1]{\rule{\linewidth}{#1}}
\setlength{\textheight}{230mm}
\allsectionsfont{ \normalfont\scshape}
\let\tmp\oddsidemargin
\let\oddsidemargin\evensidemargin
\let\evensidemargin\tmp
\reversemarginpar

\numberwithin{equation}{section} % Number equations within sections (i.e. 1.1, 1.2, 2.1, 2.2 instead of 1, 2, 3, 4)
\numberwithin{figure}{section} % Number figures within sections (i.e. 1.1, 1.2, 2.1, 2.2 instead of 1, 2, 3, 4)
\numberwithin{table}{section} % Number tables within sections (i.e. 1.1, 1.2, 2.1, 2.2 instead of 1, 2, 3, 4)

\setlength\parindent{0pt}


\begin{document}


%\sffamily

\fancypagestyle{plain}
{%
  \renewcommand{\headrulewidth}{0pt}%
  \renewcommand{\footrulewidth}{0.5pt}
  \fancyhf{}%
 \fancyhead[R]{\emph{\footnotesize \today}}
  \fancyfoot[C]{\emph{\footnotesize Boqi Chen, bochen@student.ethz.ch}\\ \emph{\footnotesize Xin Wu, xinwuxin@student.ethz.ch}}%
}



\titlehead
{
	ETH Zürich\\%
	D-ITET\\%
	Biomedical Engineering\hfill
    Master Studies%
}

\subject{\vspace{-1ex} \horrule{2pt}\\[0.15cm] {\textsc{\texttt{Biomedical Imaging}}}}

\title{Homework \#4 - Ultrasound 3\\[0.5cm]}

\author{\bfseries{Xin Wu \\ \textbf{Boqi Chen\\}}\vspace{-2ex}}
\date{\begin{tabular}{cc}
  \textsc{Date:}& \textsc{\emph{\today}}\\
  \textsc{Due :}& \textsc{\emph{19th October  2020}}\vspace{3ex}
\end{tabular}}

\maketitle

%--------------------------------------------
\newpage

\section{Task 1} 
\subsection{Question a}

According to the sampling theorem, the Doppler shifts can be resolved unambiguously where:

\begin{equation}
     f_D \in (\frac{-f_{prf}}{2},\frac{f_{prf}}{2})
\end{equation}

Thus, we obtain:

\begin{equation}
     f_D \in (-\frac{1}{2}kHz,\frac{1}{2}kHz)
\end{equation}

\subsection{Question b}
 According to the previous lecture, the relationship of the Doppler shift and the velocity of the motion along the direction of the ultrasound beam is given by the following equation:
 
\begin{equation}
     f_D = \frac{2 f_{i}\upsilon cos \theta}{c}
\end{equation}

Where $f_{i}$ is the frequency between the incident wave(i.e. the ultrasound frequency) and $\upsilon$ is the velocity of the motion along the direction of the ultrasound beam. c is the speed of the sound which is given by the following equation:

\begin{equation}
     c = \frac{Z}{\rho}
\end{equation}

Where Z is the impedance of the medium and $\rho$ is the density of the medium. 

Given the equations above, we can easily derive that:

\begin{equation}
     \upsilon = \frac{ c f_{D}}{2 f_{i} cos \theta}
\end{equation}

Let $\theta = 0$ we obtain:

\begin{equation}
     \upsilon \in (-0.125m/s, 0.125m/s)
\end{equation}

\subsection{Question c}

The spatial selectivity is achieved by focusing the beam. More specifically, when the beam is already settled for a narrower range of depth, we only work with the signal which comes from that depth. So, instead of mapping the time of the arrival of the signal proportionally to pixel(depth) index as what we do with ultrasound imaging, we gate the signal and carve out only the part that comes from the selected range of depth. 

\section{Speckle in Ultrasound Imaging}

\subsection{What is the cause of speckle noise in ultrasound imaging?}

Ultrasound imaging is a coherent imaging method like Laser Imaging. The highly coherent waves generated by scattering will interfere and create a pattern with randomly distributed bright and dark regions. 

The random speckle pattern will added to the picture as a background noise, and can't be removed by averaging over several repetitions. 

\subsection{What is the characteristic length of speckle noise?}

The typical length of speckle noise is $\frac{\lambda}{2}$.

\subsection{How can speckle noise be mitigated?}

In order to mitigate the speckle noise, the waves should be added incoherently. So compound imaging can do this work by averaging over pictures taken from different angles. Some image analysis algorithms can also mitigate the speckle noise in a picture.

\end{document}
